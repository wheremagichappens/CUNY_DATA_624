\documentclass[]{report}
\usepackage{lmodern}
\usepackage{amssymb,amsmath}
\usepackage{ifxetex,ifluatex}
\usepackage{fixltx2e} % provides \textsubscript
\ifnum 0\ifxetex 1\fi\ifluatex 1\fi=0 % if pdftex
  \usepackage[T1]{fontenc}
  \usepackage[utf8]{inputenc}
\else % if luatex or xelatex
  \ifxetex
    \usepackage{mathspec}
  \else
    \usepackage{fontspec}
  \fi
  \defaultfontfeatures{Ligatures=TeX,Scale=MatchLowercase}
\fi
% use upquote if available, for straight quotes in verbatim environments
\IfFileExists{upquote.sty}{\usepackage{upquote}}{}
% use microtype if available
\IfFileExists{microtype.sty}{%
\usepackage{microtype}
\UseMicrotypeSet[protrusion]{basicmath} % disable protrusion for tt fonts
}{}
\usepackage{hyperref}
\hypersetup{unicode=true,
            pdftitle={Homework Part 2},
            pdfauthor={Vinicio Haro; Sang Yoon (Andy) Hwang; Julian McEachern; Jeremy O'Brien; Bethany Poulin},
            pdfborder={0 0 0},
            breaklinks=true}
\urlstyle{same}  % don't use monospace font for urls
\usepackage{color}
\usepackage{fancyvrb}
\newcommand{\VerbBar}{|}
\newcommand{\VERB}{\Verb[commandchars=\\\{\}]}
\DefineVerbatimEnvironment{Highlighting}{Verbatim}{commandchars=\\\{\}}
% Add ',fontsize=\small' for more characters per line
\usepackage{framed}
\definecolor{shadecolor}{RGB}{248,248,248}
\newenvironment{Shaded}{\begin{snugshade}}{\end{snugshade}}
\newcommand{\AlertTok}[1]{\textcolor[rgb]{0.94,0.16,0.16}{#1}}
\newcommand{\AnnotationTok}[1]{\textcolor[rgb]{0.56,0.35,0.01}{\textbf{\textit{#1}}}}
\newcommand{\AttributeTok}[1]{\textcolor[rgb]{0.77,0.63,0.00}{#1}}
\newcommand{\BaseNTok}[1]{\textcolor[rgb]{0.00,0.00,0.81}{#1}}
\newcommand{\BuiltInTok}[1]{#1}
\newcommand{\CharTok}[1]{\textcolor[rgb]{0.31,0.60,0.02}{#1}}
\newcommand{\CommentTok}[1]{\textcolor[rgb]{0.56,0.35,0.01}{\textit{#1}}}
\newcommand{\CommentVarTok}[1]{\textcolor[rgb]{0.56,0.35,0.01}{\textbf{\textit{#1}}}}
\newcommand{\ConstantTok}[1]{\textcolor[rgb]{0.00,0.00,0.00}{#1}}
\newcommand{\ControlFlowTok}[1]{\textcolor[rgb]{0.13,0.29,0.53}{\textbf{#1}}}
\newcommand{\DataTypeTok}[1]{\textcolor[rgb]{0.13,0.29,0.53}{#1}}
\newcommand{\DecValTok}[1]{\textcolor[rgb]{0.00,0.00,0.81}{#1}}
\newcommand{\DocumentationTok}[1]{\textcolor[rgb]{0.56,0.35,0.01}{\textbf{\textit{#1}}}}
\newcommand{\ErrorTok}[1]{\textcolor[rgb]{0.64,0.00,0.00}{\textbf{#1}}}
\newcommand{\ExtensionTok}[1]{#1}
\newcommand{\FloatTok}[1]{\textcolor[rgb]{0.00,0.00,0.81}{#1}}
\newcommand{\FunctionTok}[1]{\textcolor[rgb]{0.00,0.00,0.00}{#1}}
\newcommand{\ImportTok}[1]{#1}
\newcommand{\InformationTok}[1]{\textcolor[rgb]{0.56,0.35,0.01}{\textbf{\textit{#1}}}}
\newcommand{\KeywordTok}[1]{\textcolor[rgb]{0.13,0.29,0.53}{\textbf{#1}}}
\newcommand{\NormalTok}[1]{#1}
\newcommand{\OperatorTok}[1]{\textcolor[rgb]{0.81,0.36,0.00}{\textbf{#1}}}
\newcommand{\OtherTok}[1]{\textcolor[rgb]{0.56,0.35,0.01}{#1}}
\newcommand{\PreprocessorTok}[1]{\textcolor[rgb]{0.56,0.35,0.01}{\textit{#1}}}
\newcommand{\RegionMarkerTok}[1]{#1}
\newcommand{\SpecialCharTok}[1]{\textcolor[rgb]{0.00,0.00,0.00}{#1}}
\newcommand{\SpecialStringTok}[1]{\textcolor[rgb]{0.31,0.60,0.02}{#1}}
\newcommand{\StringTok}[1]{\textcolor[rgb]{0.31,0.60,0.02}{#1}}
\newcommand{\VariableTok}[1]{\textcolor[rgb]{0.00,0.00,0.00}{#1}}
\newcommand{\VerbatimStringTok}[1]{\textcolor[rgb]{0.31,0.60,0.02}{#1}}
\newcommand{\WarningTok}[1]{\textcolor[rgb]{0.56,0.35,0.01}{\textbf{\textit{#1}}}}
\usepackage{graphicx,grffile}
\makeatletter
\def\maxwidth{\ifdim\Gin@nat@width>\linewidth\linewidth\else\Gin@nat@width\fi}
\def\maxheight{\ifdim\Gin@nat@height>\textheight\textheight\else\Gin@nat@height\fi}
\makeatother
% Scale images if necessary, so that they will not overflow the page
% margins by default, and it is still possible to overwrite the defaults
% using explicit options in \includegraphics[width, height, ...]{}
\setkeys{Gin}{width=\maxwidth,height=\maxheight,keepaspectratio}
\IfFileExists{parskip.sty}{%
\usepackage{parskip}
}{% else
\setlength{\parindent}{0pt}
\setlength{\parskip}{6pt plus 2pt minus 1pt}
}
\setlength{\emergencystretch}{3em}  % prevent overfull lines
\providecommand{\tightlist}{%
  \setlength{\itemsep}{0pt}\setlength{\parskip}{0pt}}
\setcounter{secnumdepth}{0}

%%% Use protect on footnotes to avoid problems with footnotes in titles
\let\rmarkdownfootnote\footnote%
\def\footnote{\protect\rmarkdownfootnote}

%%% Change title format to be more compact
\usepackage{titling}

% Create subtitle command for use in maketitle
\providecommand{\subtitle}[1]{
  \posttitle{
    \begin{center}\large#1\end{center}
    }
}

\setlength{\droptitle}{-2em}

  \title{Homework Part 2}
    \pretitle{\vspace{\droptitle}\centering\huge}
  \posttitle{\par}
    \author{Vinicio Haro \\ Sang Yoon (Andy) Hwang \\ Julian McEachern \\ Jeremy O'Brien \\ Bethany Poulin}
    \preauthor{\centering\large\emph}
  \postauthor{\par}
    \date{}
    \predate{}\postdate{}
  
% set plain style for page numbers
\pagestyle{plain}
\raggedbottom

% change font
\usepackage{fontspec}
\setmainfont{Arial}

% create color block quotes
\usepackage[dvipsnames]{xcolor}
\definecolor{navyblue}{RGB}{16, 52, 166}
\definecolor{stealblue}{RGB}{72, 90, 122}
\usepackage{tcolorbox}     
\newtcolorbox{question}[1]{colback=white, colframe=navyblue ,fonttitle=\bfseries, title=#1}

\newtcolorbox{subquestion}[1]{colback=white,colframe=white, coltitle=navyblue!75!black, detach title, before upper={\tcbtitle\quad\hangindent7mm}, title={#1},fonttitle=\bfseries, fontupper=\bfseries}


\usepackage{xcolor}
\usepackage{sectsty}
\usepackage{etoolbox}
\usepackage{titling}

\usepackage{titling}
\pretitle{\begin{flushright}\Huge\color{navyblue}\textbf}
\posttitle{\par\Large\color{gray}Data 624 - Predictive Analytics\linebreak 16 December 2019\end{flushright}}
\preauthor{\begin{flushright}\large \lineskip 0.5em\textbf{Group Members:}\linebreak\textit}
\postauthor{\par\end{flushright}}
\predate{\begin{flushright}\large}
\postdate{\par\end{flushright}}




% remove "chapter" from chapter title
\usepackage{titlesec}
\titleformat{\chapter}
  {\normalfont\color{navyblue}\LARGE\bfseries}{\thechapter}{1em}{}
\titlespacing*{\chapter}{0pt}{3.5ex plus 1ex minus .2ex}{2.3ex plus .2ex}


% margins
\usepackage[margin=1in]{geometry}

% kable 
\usepackage{tabu}
\usepackage{float} 
\usepackage{booktabs}
\usepackage[font={color=navyblue,bf}]{caption}

% multicolumn
\usepackage{multicol}
\usepackage{booktabs}
\usepackage{longtable}
\usepackage{array}
\usepackage{multirow}
\usepackage{wrapfig}
\usepackage{float}
\usepackage{colortbl}
\usepackage{pdflscape}
\usepackage{tabu}
\usepackage{threeparttable}
\usepackage{threeparttablex}
\usepackage[normalem]{ulem}
\usepackage{makecell}
\usepackage{xcolor}

\begin{document}
\maketitle

{
\setcounter{tocdepth}{2}
\tableofcontents
}
\hypertarget{Overview}{%
\chapter*{Getting Started}\label{Overview}}
\addcontentsline{toc}{chapter}{Getting Started}

\hypertarget{overview}{%
\section{Overview}\label{overview}}

Include details on our process in creating this document.

\hypertarget{dependencies}{%
\section{Dependencies}\label{dependencies}}

\begin{Shaded}
\begin{Highlighting}[]
<<<<<<< HEAD
\CommentTok{# Predicitve Modeling}
\KeywordTok{libraries}\NormalTok{(}\StringTok{"AppliedPredictiveModeling"}\NormalTok{, }\StringTok{"mice"}\NormalTok{, }\StringTok{"caret"}\NormalTok{, }\StringTok{"tidyverse"}\NormalTok{, }
    \StringTok{"impute"}\NormalTok{, }\StringTok{"pls"}\NormalTok{, }\StringTok{"caTools"}\NormalTok{, }\StringTok{"mlbench"}\NormalTok{)}
\CommentTok{# Formatting Libraries}
\KeywordTok{libraries}\NormalTok{(}\StringTok{"default"}\NormalTok{, }\StringTok{"knitr"}\NormalTok{, }\StringTok{"kableExtra"}\NormalTok{, }\StringTok{"gridExtra"}\NormalTok{, }\StringTok{"sqldf"}\NormalTok{, }
    \StringTok{"tibble"}\NormalTok{)}
\CommentTok{# Plotting Libraries}
\KeywordTok{libraries}\NormalTok{(}\StringTok{"ggplot2"}\NormalTok{, }\StringTok{"grid"}\NormalTok{, }\StringTok{"ggfortify"}\NormalTok{)}
=======
\CommentTok{# Data Wrangling}
\KeywordTok{library}\NormalTok{(AppliedPredictiveModeling)}
\KeywordTok{library}\NormalTok{(mice)}
\KeywordTok{library}\NormalTok{(caret)}
\KeywordTok{library}\NormalTok{(tidyverse)}
\KeywordTok{library}\NormalTok{(pls)}
\KeywordTok{library}\NormalTok{(caTools)}
\KeywordTok{library}\NormalTok{(mlbench)}
\KeywordTok{library}\NormalTok{(stringr)}

\CommentTok{# Formatting}
\KeywordTok{library}\NormalTok{(default)}
\KeywordTok{library}\NormalTok{(knitr)}
\KeywordTok{library}\NormalTok{(kableExtra)}

\CommentTok{# Plotting}
\KeywordTok{library}\NormalTok{(ggplot2)}
\KeywordTok{library}\NormalTok{(grid)}
\KeywordTok{library}\NormalTok{(ggfortify)}
\KeywordTok{library}\NormalTok{(gridExtra)}
>>>>>>> 637359e0cee9ea361822a3520e3358469efe2efa
\end{Highlighting}
\end{Shaded}

\newpage

\hypertarget{AS-1}{%
\chapter*{Assignment 1}\label{AS-1}}
\addcontentsline{toc}{chapter}{Assignment 1}

\addcontentsline{toc}{subsection}{Kuhn and Johnson 6.3}

\begin{question}{Kuhn and Johnson 6.3}A chemical manufacturing process for a pharmaceutical product was discussed in Sect.1.4. In this problem, the objective is to understand the relationship between biological measurements of the raw materials (predictors), measurements of the manufacturing process (predictors), and the response of product yield. Biological predictors cannot be changed but can be used to assess the quality of the raw material before processing. On the other hand, manufacturing process predictors can be changed in the manufacturing process. Improving product yield by 1\% will boost revenue by approximately one hundred thousand dollars per batch:\end{question}

\begin{subquestion}{(a).}Start R and use these commands to load the data:
\end{subquestion}

\begin{Shaded}
\begin{Highlighting}[]
\KeywordTok{data}\NormalTok{(}\StringTok{"ChemicalManufacturingProcess"}\NormalTok{)}
\end{Highlighting}
\end{Shaded}

The matrix processPredictors contains the 57 predictors (12 describing
the input biological material and 45 describing the process predictors)
for the 176 manufacturing runs. yield contains the percent yield for
each run. Using a histogram, we examined the distribution of
\texttt{Yield} and found that the response variable appears unimodal
with a normal distribution.

\begin{center}\includegraphics{Homework-Two_files/figure-latex/kj-6.3a-plot-1} \end{center}

\begin{subquestion}{(b).} A small percentage of cells in the predictor set contain missing values. Use an imputation function to fill in these missing values (e.g., see Sect. 3.8). 
\end{subquestion}

\texttt{ManufacturingProcess03} has the largest volume of missing data
followed by \texttt{ManufacturingProcess11}. Given that each variable
has less than 25\% of data missing, we should introduce methods of
imputation. For our purposes, we will use the MICE method. MICE is
formally known as Multiple Imputation with Chained Equations. On a high
level, MICE is built off a tehcnique known as the Gibbs sampler.

The Gibbs sampler is a Markov chain based on Monte Carlo. MICE iterates
drawing estimates of missing values and parameters related to the
distribution of said variables. Chained equations are generally faster
than the monte carlo based Gibbs sampler. MICE has 5 imputations listed
as its default. predictive mean matching is also a default method for
MICE. PMM does a better job at keeping non-linear relationships within
individual variables.

In addition to MICE, we drop variables that have near zero variance,
however we point out that only one variable was dropped. We still
include it as a process step to follow the literature's specifications.
After completing MICE, we no longer had missing data in our set. We
examined other imputation methods such as KNN but determined that there
was no significant change in the summary statistics across different
imputation methods.

\begin{table}[H]

\caption{\label{tab:kj-6.3b}Variables with Missing Values}
\centering
<<<<<<< HEAD
\begin{tabular}[t]{l|r|l|r}
\hline
Variable & Count & Variable & Count\\
\hline
\rowcolor{gray!6}  ManufacturingProcess03 & 15 & BiologicalMaterial01 & 0\\
\hline
ManufacturingProcess11 & 10 & BiologicalMaterial02 & 0\\
\hline
\rowcolor{gray!6}  ManufacturingProcess10 & 9 & BiologicalMaterial03 & 0\\
\hline
ManufacturingProcess25 & 5 & BiologicalMaterial04 & 0\\
\hline
\rowcolor{gray!6}  ManufacturingProcess26 & 5 & BiologicalMaterial05 & 0\\
\hline
ManufacturingProcess27 & 5 & BiologicalMaterial06 & 0\\
\hline
\rowcolor{gray!6}  ManufacturingProcess28 & 5 & BiologicalMaterial07 & 0\\
\hline
ManufacturingProcess29 & 5 & BiologicalMaterial08 & 0\\
\hline
\rowcolor{gray!6}  ManufacturingProcess30 & 5 & BiologicalMaterial09 & 0\\
\hline
ManufacturingProcess31 & 5 & BiologicalMaterial10 & 0\\
\hline
\rowcolor{gray!6}  ManufacturingProcess33 & 5 & BiologicalMaterial11 & 0\\
\hline
ManufacturingProcess34 & 5 & BiologicalMaterial12 & 0\\
\hline
\rowcolor{gray!6}  ManufacturingProcess35 & 5 & ManufacturingProcess09 & 0\\
\hline
ManufacturingProcess36 & 5 & ManufacturingProcess13 & 0\\
\hline
\rowcolor{gray!6}  ManufacturingProcess02 & 3 & ManufacturingProcess15 & 0\\
\hline
ManufacturingProcess06 & 2 & ManufacturingProcess16 & 0\\
\hline
\rowcolor{gray!6}  ManufacturingProcess01 & 1 & ManufacturingProcess17 & 0\\
\hline
ManufacturingProcess04 & 1 & ManufacturingProcess18 & 0\\
\hline
\rowcolor{gray!6}  ManufacturingProcess05 & 1 & ManufacturingProcess19 & 0\\
\hline
ManufacturingProcess07 & 1 & ManufacturingProcess20 & 0\\
\hline
\rowcolor{gray!6}  ManufacturingProcess08 & 1 & ManufacturingProcess21 & 0\\
\hline
ManufacturingProcess12 & 1 & ManufacturingProcess32 & 0\\
\hline
\rowcolor{gray!6}  ManufacturingProcess14 & 1 & ManufacturingProcess37 & 0\\
\hline
ManufacturingProcess22 & 1 & ManufacturingProcess38 & 0\\
\hline
\rowcolor{gray!6}  ManufacturingProcess23 & 1 & ManufacturingProcess39 & 0\\
\hline
ManufacturingProcess24 & 1 & ManufacturingProcess42 & 0\\
\hline
\rowcolor{gray!6}  ManufacturingProcess40 & 1 & ManufacturingProcess43 & 0\\
\hline
ManufacturingProcess41 & 1 & ManufacturingProcess44 & 0\\
\hline
=======
\fontsize{8}{10}\selectfont
\begin{tabular}{lr>{\bfseries\raggedright\arraybackslash}p{0.1cm}lr}
\toprule
\textbf{Predictor} & \textbf{n} & \textbf{ } & \textbf{Predictor} & \textbf{n}\\
\midrule
\rowcolor{gray!6}  ManufacturingProcess03 & 15 &  & ManufacturingProcess02 & 3\\
ManufacturingProcess11 & 10 &  & ManufacturingProcess06 & 2\\
\rowcolor{gray!6}  ManufacturingProcess10 & 9 &  & ManufacturingProcess01 & 1\\
ManufacturingProcess25 & 5 &  & ManufacturingProcess04 & 1\\
\rowcolor{gray!6}  ManufacturingProcess26 & 5 &  & ManufacturingProcess05 & 1\\
\addlinespace
ManufacturingProcess27 & 5 &  & ManufacturingProcess07 & 1\\
\rowcolor{gray!6}  ManufacturingProcess28 & 5 &  & ManufacturingProcess08 & 1\\
ManufacturingProcess29 & 5 &  & ManufacturingProcess12 & 1\\
\rowcolor{gray!6}  ManufacturingProcess30 & 5 &  & ManufacturingProcess14 & 1\\
ManufacturingProcess31 & 5 &  & ManufacturingProcess22 & 1\\
\addlinespace
\rowcolor{gray!6}  ManufacturingProcess33 & 5 &  & ManufacturingProcess23 & 1\\
ManufacturingProcess34 & 5 &  & ManufacturingProcess24 & 1\\
\rowcolor{gray!6}  ManufacturingProcess35 & 5 &  & ManufacturingProcess40 & 1\\
ManufacturingProcess36 & 5 &  & ManufacturingProcess41 & 1\\
\bottomrule
>>>>>>> 637359e0cee9ea361822a3520e3358469efe2efa
\end{tabular}
\end{table}

\begin{subquestion}{(c).} Split the data into a training and a test set, pre-process the data, and tune a model of your choice from this chapter. What is the optimal value of the performance metric? 
\end{subquestion}

We will build a PLS model also known as partial least squares. PLS is a
statistical method that fits a linear regression model by projecting the
feature variables and response variable to some new space via a mapping
function. Because of this projection mechanisim, for both predictors and
the response, the method becomes bilinear or simply known as linear with
respect to to each of the variable types. PLS also has certain
advantages over other methods such as being more robust to dealing with
issues arising from multicolinearity.

For our PLS model, we partitioned the data by taking 80\% of the data as
training and the remaining 20\% as testing subsets. We also apply center
and scaling arguments set to true. We built a standard PLS model and
evaluated the root mean summary areas to determine the optimal number of
components to select. We generate performance metrics for our best tune
below:

\begin{table}[H]

\caption{\label{tab:kj-6.3c}PLS Performance Metrics on Training Subset}
\centering
<<<<<<< HEAD
\begin{tabular}[t]{l|r}
\hline
  & x\\
\hline
\rowcolor{gray!6}  RMSE & 1.3116314\\
\hline
Rsquared & 0.4896581\\
\hline
\rowcolor{gray!6}  MAE & 1.0584889\\
\hline
=======
\fontsize{8}{10}\selectfont
\begin{tabular}{rrr}
\toprule
\textbf{RMSE} & \textbf{Rsquared} & \textbf{MAE}\\
\midrule
\rowcolor{gray!6}  1.5626 & 0.3953 & 1.1845\\
\bottomrule
>>>>>>> 637359e0cee9ea361822a3520e3358469efe2efa
\end{tabular}
\end{table}

Our Baseline PLS model generates a RMSE of 1.56. In addition, the model
captures 39.53 \% of data variability. We include the visualizations
pertaining to the train set cross-validation RMSE tunes and a plot
comparing the observed and predicted outcome from our model.

\includegraphics{Homework-Two_files/figure-latex/kj-6.3c2-1.pdf}

\begin{subquestion}{(d).} Predict the response for the test set. What is the value of the performance metric and how does this compare with the resampled performance metric on the training set? 
\end{subquestion}

We see a decreased R squared against the test data with 22\% of the data
variability accounted for. We also see the RMSE decrease to 1.52 from
our training results of 1.56. There is also a slight increase in the
MAE.

\begin{table}[H]

\caption{\label{tab:kj-6.3d-1}PLS Performance Metrics on Test Subset}
\centering
<<<<<<< HEAD
\begin{tabular}[t]{l|r}
\hline
  & x\\
\hline
\rowcolor{gray!6}  RMSE & 1.222201\\
\hline
Rsquared & 0.579859\\
\hline
\rowcolor{gray!6}  MAE & 0.988292\\
\hline
=======
\fontsize{8}{10}\selectfont
\begin{tabular}{rrr}
\toprule
\textbf{RMSE} & \textbf{Rsquared} & \textbf{MAE}\\
\midrule
\rowcolor{gray!6}  1.5222 & 0.2212 & 1.2885\\
\bottomrule
>>>>>>> 637359e0cee9ea361822a3520e3358469efe2efa
\end{tabular}
\end{table}

We also plotted the observed and predicted values from our test set
against each other below. The deviation from the fitted line tells us
that our selected linear model may not provide the best predictions for
\texttt{Yield}.

\begin{center}\includegraphics{Homework-Two_files/figure-latex/kj-6.3d-2-1} \end{center}

\begin{subquestion}{(e).} Which predictors are most important in the model you have trained? Do either the biological or process predictors dominate the list? 
\end{subquestion}

VarImp allows us to identify the variables by name and compute their
importance. \texttt{ManufacturingProcess32} was flagged as the most
important predictor overall and within the group of other Manufacturing
Process variables. \texttt{BiologicalMaterial06} ranked second and was
the most important variable within the BiologicalMaterial group. The
variable importance rankings are mixed with 8 variables belonging to
Biology and 7 Manufacturing Process variables within the top 15
predictors.

\includegraphics{Homework-Two_files/figure-latex/kj-6.3e-1.pdf}

\begin{subquestion}{(f).} Explore the relationships between each of the top predictors and the response. How could this information be helpful in improving yield in future runs of the manufacturing process?
\end{subquestion}

We used a scatter plot to visualize the relationship between our top
five important predictors against our response variable, \texttt{Yield}.
All but \texttt{ManufacturingProcess32} show a moderate positive, linear
relationship with yield.

\includegraphics{Homework-Two_files/figure-latex/kj-6.3f-1-1.pdf}

We further examined this relationship by analyzing the correlation
strength between our top five important response variables with the
\texttt{Yield}. Out of which, \texttt{ManufacturingProcess32} showed the
strongest, positive correlation with our response variable.

From a business point of view, our aim is to increase yield since we
know that yield ties into revenue. We do not have insight into what
mechanics go into each manufacturing process but we can use this
knowledge to adjust the processes to emulate the highest yield outputs.

\begin{table}[H]

\caption{\label{tab:kj-6.3f-2}Variable Correlation with Yield}
\centering
<<<<<<< HEAD
\begin{tabular}[t]{l|r}
\hline
  & Yield\\
\hline
\rowcolor{gray!6}  Yield & 1.0000000\\
\hline
ManufacturingProcess32 & 0.6083321\\
\hline
\rowcolor{gray!6}  ManufacturingProcess17 & -0.4258069\\
\hline
ManufacturingProcess13 & -0.5036797\\
\hline
\rowcolor{gray!6}  ManufacturingProcess36 & -0.4907962\\
\hline
ManufacturingProcess09 & 0.5034705\\
\hline
=======
\fontsize{8}{10}\selectfont
\begin{tabular}{lr}
\toprule
\textbf{Variable} & \textbf{Yield}\\
\midrule
\rowcolor{gray!6}  ManufacturingProcess32 & 0.6083\\
BiologicalMaterial02 & 0.4815\\
\rowcolor{gray!6}  BiologicalMaterial06 & 0.4782\\
BiologicalMaterial03 & 0.4451\\
\rowcolor{gray!6}  ManufacturingProcess36 & -0.5014\\
\bottomrule
>>>>>>> 637359e0cee9ea361822a3520e3358469efe2efa
\end{tabular}
\end{table}

\hypertarget{AS-2}{%
\chapter*{Assignment 2}\label{AS-2}}
\addcontentsline{toc}{chapter}{Assignment 2}

\addcontentsline{toc}{subsection}{Kuhn and Johnson 7.2}

\begin{question}{Kuhn and Johnson 7.2}Friedman (1991) introduced several benchmark data sets create by simulation. One of these simulations used the following nonlinear equation to create data: $y = 10\text{sin}(\pi x_1 x_2)+20(x_3-0.5)^2+10x_4+5x_5+N(0\text{,} \sigma^2)$; where the $x$ values are random variables uniformly distributed between $[0, 1]$ (there are also 5 other non-informative variables also created in the simulation). 
\newline
The package `mlbench` contains a function called `mlbench.friedman1` that simulates these data. We convert the 'x' data from a matrix to a data frame. One reason is that this will give the columns names. The `testData` code creates a list with a vector 'y' and a matrix of predictors 'x'. It also simulates a large test set to estimate the true error rate with good precision: \end{question}

\begin{Shaded}
\begin{Highlighting}[]
\KeywordTok{set.seed}\NormalTok{(}\DecValTok{200}\NormalTok{)}
\NormalTok{trainingData <-}\StringTok{ }\KeywordTok{mlbench.friedman1}\NormalTok{(}\DecValTok{200}\NormalTok{, }\DataTypeTok{sd =} \DecValTok{1}\NormalTok{)}
\NormalTok{trainingData}\OperatorTok{$}\NormalTok{x <-}\StringTok{ }\KeywordTok{data.frame}\NormalTok{(trainingData}\OperatorTok{$}\NormalTok{x)}
\NormalTok{testData <-}\StringTok{ }\KeywordTok{mlbench.friedman1}\NormalTok{(}\DecValTok{5000}\NormalTok{, }\DataTypeTok{sd =} \DecValTok{1}\NormalTok{)}
\NormalTok{testData}\OperatorTok{$}\NormalTok{x <-}\StringTok{ }\KeywordTok{data.frame}\NormalTok{(testData}\OperatorTok{$}\NormalTok{x)}
\end{Highlighting}
\end{Shaded}

\includegraphics{Homework-Two_files/figure-latex/kj-7.2-ex3-1.pdf}

\begin{subquestion}{(a).} Tune several models on these data. For example: 
\end{subquestion}

<<<<<<< HEAD
Model 1 MARS Regression: MARS, otherwise known as multivariate adapitive
regression splines is a non parametric regression technique that
automatically captures non-linearity and interaction between predictors.
The basic MARS model has the following form:
=======
\begin{Shaded}
\begin{Highlighting}[]
\NormalTok{knnModel <-}\StringTok{ }\KeywordTok{train}\NormalTok{(}\DataTypeTok{x =}\NormalTok{ trainingData}\OperatorTok{$}\NormalTok{x, }\DataTypeTok{y =}\NormalTok{ trainingData}\OperatorTok{$}\NormalTok{y, }\DataTypeTok{method =} \StringTok{"knn"}\NormalTok{, }
    \DataTypeTok{preProc =} \KeywordTok{c}\NormalTok{(}\StringTok{"center"}\NormalTok{, }\StringTok{"scale"}\NormalTok{), }\DataTypeTok{tuneLength =} \DecValTok{10}\NormalTok{)}
\NormalTok{knnModel}
\NormalTok{knnPred <-}\StringTok{ }\KeywordTok{predict}\NormalTok{(knnModel, }\DataTypeTok{newdata =}\NormalTok{ testData}\OperatorTok{$}\NormalTok{x)}
\KeywordTok{postResample}\NormalTok{(}\DataTypeTok{pred =}\NormalTok{ knnPred, }\DataTypeTok{obs =}\NormalTok{ testData}\OperatorTok{$}\NormalTok{y)}
\end{Highlighting}
\end{Shaded}

\#\#\#Model 1:
>>>>>>> 637359e0cee9ea361822a3520e3358469efe2efa

\[
\overset { \wedge  }{ f } =\sum _{ i=1 }^{ k }{ { c }_{ i }{ B }_{ i }(x) } 
\]

The model computes the sum of basis functions B multiplied by constant
coefficients c.The basis function can either be a constant, a hinge
function, or a product of hinge functions. By definition, a hinge
function is a piecewise function that converges at a point known as a
knot.

\includegraphics{Homework-Two_files/figure-latex/kj-7.2-1-1.pdf}

Model 2 SVM: SVM, also known as support vector machine is a method that
can be applied to classification and regression tasks. On a high level,
SVM creates a hyperplane in n dimensional space. This hyperplane acts
like a classification boundary which can be linear or non linear. This
boundary classifies information from a feature space.

\includegraphics{Homework-Two_files/figure-latex/kj-7.2-2-1.pdf}

Model 3 NNET:

NNEt otherwise known as a Neural Network, is a method inspired by a
biological neuron system. It uses a system of nodes that are parallel to
the way neurons work. It is ideal for capturing non-linear relationships
that would otherwise be complicated in a multiple linear regression
model. NNET evolves internally based on the calculated weights of each
input. The basic structure is shown below:

\[
Y=\sum { (weight\quad *\quad input)\quad +\quad bias } 
\]

\includegraphics{Homework-Two_files/figure-latex/kj-7.2-3-1.pdf}

\begin{subquestion}{(b).}
Which models appear to give the best performance? Does MARS select the informative predictors (those named X1-X5)?
\end{subquestion}

\begin{table}[H]

\caption{\label{tab:unnamed-chunk-1}Model Performance}
\centering
\begin{tabular}[t]{l|r|r|r}
\hline
\textbf{ } & \textbf{RMSE} & \textbf{RSquared} & \textbf{MAE}\\
\hline
\rowcolor{gray!6}  knnTrain & 3.6521 & 3.6521 & 3.6521\\
\hline
knnTest & 3.2041 & 0.6820 & 2.5683\\
\hline
\rowcolor{gray!6}  MARSTrain & 4.4548 & 0.9389 & 3.7461\\
\hline
MARSTest & 1.1723 & 0.9449 & 0.9325\\
\hline
\rowcolor{gray!6}  SVMTrain & 2.4940 & 0.8561 & 1.9931\\
\hline
SVMTest & 2.0699 & 0.8263 & 1.5723\\
\hline
\rowcolor{gray!6}  \rowcolor[HTML]{d9f2e6}  \textbf{NNETTrain} & \textbf{13.7411} & \textbf{0.8021} & \textbf{5.5596}\\
\hline
\rowcolor[HTML]{d9f2e6}  \textbf{NNETTest} & \textbf{2.5669} & \textbf{0.7427} & \textbf{1.9506}\\
\hline
\end{tabular}
\end{table}

MARS appears to give the best performance based on RMSE, R squared and
MAE on test set. The above table shows how our other selected models
stack up against the best performing MARS model. We now evaluate the
variable importance for our best performing model.

\includegraphics{Homework-Two_files/figure-latex/kj-7.2-4b-1.pdf}

The variable importance table for MARS indicates that variables X1
through X5 were picked as the most important. Out of our collection of
important variables used in MARS, X1 is the most important.

It is very likely that the lack of contribution allotted to the X6-X10
variables which bolster the R Squared and RMSE performance and noise
from these variables did not reduce the predictive strength of this
model as it does in small quantities in the other three models.

\addcontentsline{toc}{subsection}{Kuhn and Johnson 7.5}

\begin{question}{Kuhn and Johnson 7.5}
Exercise 6.3 describes data for a chemical manufacturing process. Use the same data imputation, data splitting, and pre-processing steps as before and train several nonlinear regression models.
\end{question}

We pulled in the data processing method from 6.3. This includes
imputation and removal of zero processing. Please refer to 6.3 for a
more detailed look at the EDA involved with this data set. We tunned a
KNN model, NNET model, MARS, and SVM model using specifications from the
literature.

\begin{subquestion}{(a).}
Which nonlinear regression model gives the optimal resampling and test set performance? 
\end{subquestion}

\begin{table}[H]

\caption{\label{tab:unnamed-chunk-1}Model Performance on ChemicalManufacturing Data}
\centering
\begin{tabular}[t]{l|r|r|r}
\hline
\textbf{ } & \textbf{RMSE} & \textbf{RSquared} & \textbf{MAE}\\
\hline
\rowcolor{gray!6}  knnTrain & 1.4835 & 0.4245 & 1.1746\\
\hline
knnTest & 1.4078 & 0.4477 & 1.1041\\
\hline
\rowcolor{gray!6}  MARSTrain & 1.6367 & 0.6488 & 1.1614\\
\hline
MARSTest & 1.3726 & 0.4682 & 1.1300\\
\hline
\rowcolor{gray!6}  \rowcolor[HTML]{d9f2e6}  \textbf{SVMTrain} & \textbf{1.4142} & \textbf{0.7144} & \textbf{1.1448}\\
\hline
\rowcolor[HTML]{d9f2e6}  \textbf{SVMTest} & \textbf{1.2061} & \textbf{0.5715} & \textbf{0.9833}\\
\hline
\rowcolor{gray!6}  NNETTrain & 9.9082 & 0.3704 & 6.2548\\
\hline
NNETTest & 1.4683 & 0.3987 & 1.1936\\
\hline
\end{tabular}
\end{table}

Radial SVM outperformed the other models across all key KPI's. The next
best model was MARS. In part b, we will address what variables are
dominant in our SVM model.

\begin{subquestion}{(b).}
Which predictors are most important in the optimal nonlinear regression model? Do either the biological or process variables dominate the list? How do the top ten important predictors compare to the top ten predictors from the optimal linear model? 
\end{subquestion}

\includegraphics{Homework-Two_files/figure-latex/kj-7.5b-1.pdf}

\texttt{ManufacturingProcess} Variables dominate the ranking of
important variables with ManufacturingProcess14 at the top.
ManufacturingProcess32 was at the top of the important variables list
when it came to our linear model with some Biological Process within the
top 10.

\begin{subquestion}{(c).}
Explore the relationships between the top predictors and the response for the predictors that are unique to the optimal nonlinear regression model. Do these plots reveal intuition about the biological or process predictors and their relationship with yield?
\end{subquestion}

\begin{table}

\caption{\label{tab:unnamed-chunk-1}Correlation}
\centering
\resizebox{\linewidth}{!}{
\begin{tabular}[t]{l|r}
\hline
VALUE & Yield\\
\hline
Yield & 1.0000000\\
\hline
ManufacturingProcess14 & -0.0099574\\
\hline
ManufacturingProcess38 & -0.0864593\\
\hline
ManufacturingProcess03 & -0.1189520\\
\hline
ManufacturingProcess37 & -0.1593141\\
\hline
ManufacturingProcess02 & -0.1953977\\
\hline
\end{tabular}}
\end{table}

We examined the top 5 predictors that were flagged as being the most
important before the importance measure dropped.There are some pretty
clear differences in the data which might explain both the overall poor
performance of the linear models as well as the improved significance of
Process-Based variables in the non-linear models.

Of the \texttt{ManfuacturingProcess} variables, they appear to be either
tight clusters or discrete values which predict an array of possible
Yields, which is directly opposed the the definition of linearly
separable data base on earlier examination of correlation plots.

\hypertarget{AS-3}{%
\chapter*{Assignment 3}\label{AS-3}}
\addcontentsline{toc}{chapter}{Assignment 3}

\addcontentsline{toc}{subsection}{Kuhn and Johnson 8.1}

\begin{question}{Kuhn and Johnson 8.1} Recreate the simulated data from Exercise 7.2: \end{question}

\begin{subquestion}{(a).} Fit a random forest model to all of the predictors, then estimate the variable importance scores. Did the random forest model significantly use the uninformative predictors (V6-V10)?\end{subquestion}

\begin{subquestion}{(b).} Now add an additional predictor that is highly correlated with one of the informative predictors. Fit another random forest model to these data. Did the importance score for V1 change? What happens when you add another predictor that is also highly correlated with V1? For example:\end{subquestion}

\begin{subquestion}{(c).} Use the `cforest` function in the party package to fit a random forest model using conditional inference trees. The party package function `varimp` can calculate predictor importance. The `conditional` argument of that function toggles between the traditional importance measure and the modified version described in Strobl et al. (2007). Do these importances show the same pattern as the traditional random forest model?\end{subquestion}

\begin{subquestion}{(d).} Repeat this process with different tree models, such as boosted trees and Cubist. Does the same pattern occur?\end{subquestion}

\addcontentsline{toc}{subsection}{Kuhn and Johnson 8.2}

\begin{question}{Kuhn and Johnson 8.2}Use a simulation to show tree bias with different granularities.\end{question}

\addcontentsline{toc}{subsection}{Kuhn and Johnson 8.3}

\begin{question}{Kuhn and Johnson 8.3} In stochastic gradient boosting the bagging fraction and learning rate will govern the construction of the trees as they are guided by the gradient. Although the optimal values of these parameters should be obtained through the tuning process, it is helpful to understand how the magnitudes of these parameters affect magnitudes of variable importance. Figure 8.24 provides the variable importance plots for boosting using two extreme values for the bagging fraction (0.1 and 0.9) and the learning rate (0.1 and 0.9) for the solubility data. The left-hand plot has both parameters set to 0.1, and the right-hand plot has both set to 0.9: \end{question}

\begin{subquestion}{(a).} Why does the model on the right focus its importance on just the first few of predictors, whereas the model on the left spreads importance across more predictors? \end{subquestion}

\begin{subquestion}{(b).} Which model do you think would be more predictive of other samples?\end{subquestion}

\begin{subquestion}{(c).} How would increasing interaction depth affect the slope of predictor importance for either model in Fig.8.24?\end{subquestion}

\addcontentsline{toc}{subsection}{Kuhn and Johnson 8.7}

\begin{question}{Kuhn and Johnson 8.7}
Refer to Exercises 6.3 and 7.5 which describe a chemical manufacturing process. Use the same data imputation, data splitting, and pre-processing steps as before and train several tree-based models:
\end{question}

\begin{subquestion}{(a).} Which tree-based regression model gives the optimal resampling and test set performance? \end{subquestion}

\begin{subquestion}{(b).} Which predictors are most important in the optimal tree-based regression model? Do either the biological or process variables dominate the list? How do the top 10 important predictors compare to the top 10 predictors from the optimal linear and nonlinear models?\end{subquestion}

\begin{subquestion}{(c).} Plot the optimal single tree with the distribution of yield in the terminal nodes. Does this view of the data provide additional knowledge about the biological or process predictors and their relationship with yield?\end{subquestion}

\hypertarget{AS-4}{%
\chapter*{Assignment 4}\label{AS-4}}
\addcontentsline{toc}{chapter}{Assignment 4}

\hypertarget{tbd}{%
\section{TBD}\label{tbd}}

\hypertarget{R-Script}{%
\chapter*{R Script}\label{R-Script}}
\addcontentsline{toc}{chapter}{R Script}


\end{document}
