\documentclass[]{report}
\usepackage{lmodern}
\usepackage{amssymb,amsmath}
\usepackage{ifxetex,ifluatex}
\usepackage{fixltx2e} % provides \textsubscript
\ifnum 0\ifxetex 1\fi\ifluatex 1\fi=0 % if pdftex
  \usepackage[T1]{fontenc}
  \usepackage[utf8]{inputenc}
\else % if luatex or xelatex
  \ifxetex
    \usepackage{mathspec}
  \else
    \usepackage{fontspec}
  \fi
  \defaultfontfeatures{Ligatures=TeX,Scale=MatchLowercase}
\fi
% use upquote if available, for straight quotes in verbatim environments
\IfFileExists{upquote.sty}{\usepackage{upquote}}{}
% use microtype if available
\IfFileExists{microtype.sty}{%
\usepackage{microtype}
\UseMicrotypeSet[protrusion]{basicmath} % disable protrusion for tt fonts
}{}
\usepackage{hyperref}
\hypersetup{unicode=true,
            pdftitle={DATA 624: Project 2},
            pdfauthor={Vinicio Haro; Sang Yoon (Andy) Hwang; Julian McEachern; Jeremy O'Brien; Bethany Poulin},
            pdfborder={0 0 0},
            breaklinks=true}
\urlstyle{same}  % don't use monospace font for urls
\usepackage{color}
\usepackage{fancyvrb}
\newcommand{\VerbBar}{|}
\newcommand{\VERB}{\Verb[commandchars=\\\{\}]}
\DefineVerbatimEnvironment{Highlighting}{Verbatim}{commandchars=\\\{\}}
% Add ',fontsize=\small' for more characters per line
\usepackage{framed}
\definecolor{shadecolor}{RGB}{248,248,248}
\newenvironment{Shaded}{\begin{snugshade}}{\end{snugshade}}
\newcommand{\AlertTok}[1]{\textcolor[rgb]{0.94,0.16,0.16}{#1}}
\newcommand{\AnnotationTok}[1]{\textcolor[rgb]{0.56,0.35,0.01}{\textbf{\textit{#1}}}}
\newcommand{\AttributeTok}[1]{\textcolor[rgb]{0.77,0.63,0.00}{#1}}
\newcommand{\BaseNTok}[1]{\textcolor[rgb]{0.00,0.00,0.81}{#1}}
\newcommand{\BuiltInTok}[1]{#1}
\newcommand{\CharTok}[1]{\textcolor[rgb]{0.31,0.60,0.02}{#1}}
\newcommand{\CommentTok}[1]{\textcolor[rgb]{0.56,0.35,0.01}{\textit{#1}}}
\newcommand{\CommentVarTok}[1]{\textcolor[rgb]{0.56,0.35,0.01}{\textbf{\textit{#1}}}}
\newcommand{\ConstantTok}[1]{\textcolor[rgb]{0.00,0.00,0.00}{#1}}
\newcommand{\ControlFlowTok}[1]{\textcolor[rgb]{0.13,0.29,0.53}{\textbf{#1}}}
\newcommand{\DataTypeTok}[1]{\textcolor[rgb]{0.13,0.29,0.53}{#1}}
\newcommand{\DecValTok}[1]{\textcolor[rgb]{0.00,0.00,0.81}{#1}}
\newcommand{\DocumentationTok}[1]{\textcolor[rgb]{0.56,0.35,0.01}{\textbf{\textit{#1}}}}
\newcommand{\ErrorTok}[1]{\textcolor[rgb]{0.64,0.00,0.00}{\textbf{#1}}}
\newcommand{\ExtensionTok}[1]{#1}
\newcommand{\FloatTok}[1]{\textcolor[rgb]{0.00,0.00,0.81}{#1}}
\newcommand{\FunctionTok}[1]{\textcolor[rgb]{0.00,0.00,0.00}{#1}}
\newcommand{\ImportTok}[1]{#1}
\newcommand{\InformationTok}[1]{\textcolor[rgb]{0.56,0.35,0.01}{\textbf{\textit{#1}}}}
\newcommand{\KeywordTok}[1]{\textcolor[rgb]{0.13,0.29,0.53}{\textbf{#1}}}
\newcommand{\NormalTok}[1]{#1}
\newcommand{\OperatorTok}[1]{\textcolor[rgb]{0.81,0.36,0.00}{\textbf{#1}}}
\newcommand{\OtherTok}[1]{\textcolor[rgb]{0.56,0.35,0.01}{#1}}
\newcommand{\PreprocessorTok}[1]{\textcolor[rgb]{0.56,0.35,0.01}{\textit{#1}}}
\newcommand{\RegionMarkerTok}[1]{#1}
\newcommand{\SpecialCharTok}[1]{\textcolor[rgb]{0.00,0.00,0.00}{#1}}
\newcommand{\SpecialStringTok}[1]{\textcolor[rgb]{0.31,0.60,0.02}{#1}}
\newcommand{\StringTok}[1]{\textcolor[rgb]{0.31,0.60,0.02}{#1}}
\newcommand{\VariableTok}[1]{\textcolor[rgb]{0.00,0.00,0.00}{#1}}
\newcommand{\VerbatimStringTok}[1]{\textcolor[rgb]{0.31,0.60,0.02}{#1}}
\newcommand{\WarningTok}[1]{\textcolor[rgb]{0.56,0.35,0.01}{\textbf{\textit{#1}}}}
\usepackage{graphicx}
% grffile has become a legacy package: https://ctan.org/pkg/grffile
\IfFileExists{grffile.sty}{%
\usepackage{grffile}
}{}
\makeatletter
\def\maxwidth{\ifdim\Gin@nat@width>\linewidth\linewidth\else\Gin@nat@width\fi}
\def\maxheight{\ifdim\Gin@nat@height>\textheight\textheight\else\Gin@nat@height\fi}
\makeatother
% Scale images if necessary, so that they will not overflow the page
% margins by default, and it is still possible to overwrite the defaults
% using explicit options in \includegraphics[width, height, ...]{}
\setkeys{Gin}{width=\maxwidth,height=\maxheight,keepaspectratio}
\IfFileExists{parskip.sty}{%
\usepackage{parskip}
}{% else
\setlength{\parindent}{0pt}
\setlength{\parskip}{6pt plus 2pt minus 1pt}
}
\setlength{\emergencystretch}{3em}  % prevent overfull lines
\providecommand{\tightlist}{%
  \setlength{\itemsep}{0pt}\setlength{\parskip}{0pt}}
\setcounter{secnumdepth}{0}

%%% Use protect on footnotes to avoid problems with footnotes in titles
\let\rmarkdownfootnote\footnote%
\def\footnote{\protect\rmarkdownfootnote}

%%% Change title format to be more compact
\usepackage{titling}

% Create subtitle command for use in maketitle
\providecommand{\subtitle}[1]{
  \posttitle{
    \begin{center}\large#1\end{center}
    }
}

\setlength{\droptitle}{-2em}

  \title{DATA 624: Project 2}
    \pretitle{\vspace{\droptitle}\centering\huge}
  \posttitle{\par}
    \author{Vinicio Haro \\ Sang Yoon (Andy) Hwang \\ Julian McEachern \\ Jeremy O'Brien \\ Bethany Poulin}
    \preauthor{\centering\large\emph}
  \postauthor{\par}
      \predate{\centering\large\emph}
  \postdate{\par}
    \date{10 December 2019}

% set plain style for page numbers
\usepackage[margin=1in]{geometry}
\usepackage{fancyhdr}
\pagestyle{fancy}
\fancyhead[LE,RO]{\textbf{Group 2}}
\fancyhead[RE,LO]{\textbf{Project 2: Predicting PH}}
\raggedbottom
\setlength{\parskip}{1em}

% change font
\usepackage{fontspec}
\setmainfont{Arial}

% format titles 
\usepackage{xcolor}
\usepackage{sectsty}
\usepackage{etoolbox}
\usepackage{titling}
\definecolor{prettyblue}{RGB}{84, 144, 240}
\definecolor{bluegray}{RGB}{98, 107, 115}
\pretitle{\begin{center}\Huge\color{prettyblue}\textbf}
\posttitle{\par\LARGE\color{gray}DATA 624 - Predictive Analytics\linebreak Group 2\end{center}}
\preauthor{\begin{center}\large\textbf{Group Members:}\linebreak\textit}
\postauthor{\end{center}}

% Format chapter output
\usepackage{titlesec}
\titleclass{\part}{top}
\titleclass{\chapter}{straight}
\titleformat{\chapter}
  {\normalfont\color{prettyblue}\LARGE\bfseries}{\thechapter}{1em}{}
\titlespacing*{\chapter}{0pt}{3.5ex plus 1ex minus .2ex}{2.3ex plus .2ex}


% create color block quotes
\usepackage{tcolorbox}
\newtcolorbox{myquote}{colback=purple!05!white, colframe=purple!75!black}
\renewenvironment{quote}{\begin{myquote}}{\end{myquote}}

% kable 
\usepackage{tabu}


% multicolumn
\usepackage{multicol}

% bullets
\newenvironment{tight_enumerate}{
\begin{enumerate}
  \setlength{\itemsep}{0pt}
  \setlength{\parskip}{0pt}
  }{\end{enumerate}}
  
\newenvironment{tight_itemize}{
\begin{itemize}
  \setlength{\topsep}{0pt}
  \setlength{\itemsep}{0pt}
  \setlength{\parskip}{0pt}
  \setlength{\parsep}{0pt}
  }{\end{itemize}}

\usepackage{paralist}

%hyperlink
\usepackage{hyperref}
\hypersetup{
    colorlinks=true,
    linkcolor=bluegray,
    filecolor=magenta,      
    urlcolor=cyan}

\usepackage{graphicx}
\usepackage{wrapfig}
\usepackage{booktabs}
\definecolor{yale}{RGB}{13,77,146}
\usepackage[font={color=yale,bf,scriptsize},figurename=Fig.,belowskip=0pt,aboveskip=0pt]{caption}
\usepackage{floatrow}
\floatsetup[figure]{capposition=above}
\floatsetup[table]{capposition=above}
\setlength{\abovecaptionskip}{1pt}
\setlength{\belowcaptionskip}{1pt}
\setlength{\textfloatsep}{2pt plus 0.5pt minus 0.5pt}
\setlength{\intextsep}{2pt plus 0.5pt minus 0.5pt}
\usepackage{booktabs}
\usepackage{longtable}
\usepackage{array}
\usepackage{multirow}
\usepackage{wrapfig}
\usepackage{float}
\usepackage{colortbl}
\usepackage{pdflscape}
\usepackage{tabu}
\usepackage{threeparttable}
\usepackage{threeparttablex}
\usepackage[normalem]{ulem}
\usepackage{makecell}
\usepackage{xcolor}

\begin{document}
\maketitle

{
\setcounter{tocdepth}{1}
\tableofcontents
}
File for final submission of Project 2.

\thispagestyle{empty}
\newpage
\clearpage
\pagenumbering{arabic}

\hypertarget{will-update-sunday-morning}{%
\chapter{Will Update Sunday Morning}\label{will-update-sunday-morning}}

{[}CONTENT EDITORS: ADD IN SUMMARY OF CONCLUSION AND ANY SUPPORTING
INSIGHT FROM LAST SECTION{]}

\hypertarget{introduction}{%
\chapter{Introduction}\label{introduction}}

pH is a central component to the manufacturing of a commercial beverage
as it is an indicator of both the process health and the ultimate flavor
appeal of the final product. In fact pH plays a role in multiple facets
of the a drinks appeal. The flavor, mouthfeel and the aesthetic
experience of a given product is distinctly tied to the pH relative to
other beverage qualities that brands use distinguish themselves from
other liquid refreshments.

Because it is to central to the design of a product's drinking
experience, pH is a key performance indicator in the beverage
manufacturing process and is tested for and tracked diligently, as the
final pH is dependent on and vulnerable to even slight changes in
production methods.

Having monitored and recorded these production variables, as well as the
final pH, we have the opportunity to improve production outcomes by more
closely controlling pH in our beverages with predictive modeling with
the potential to catch and correct variations in process which negative
impact our taget pH.

Each group member worked individually to experiment with preprocessing
while exploring a distinct set of model methods. Upon review, our team
created a singular preprocessing protocol and data set, based on our
most successful methods and evaluated our most performant models built
over these data.

\hypertarget{links-to-work-product-lame-title---do-we-keep-this-section}{%
\subsubsection{Links to Work Product {[}lame title - do we keep this
section?{]}}\label{links-to-work-product-lame-title---do-we-keep-this-section}}

\begin{compactitem}
  \item \href{https://github.com/JeremyOBrien16/CUNY_DATA_624/tree/master/Project_Two}{Pretend I'm a working link to R Source Code}
  \item \href{https://github.com/JeremyOBrien16/CUNY_DATA_624/tree/master/Project_Two}{Pretend I'm a working link to Provided Data}
  \item \href{https://github.com/JeremyOBrien16/CUNY_DATA_624/tree/master/Project_Two}{Pretend I'm a working link to Excel Results}
  \item \href{https://github.com/JeremyOBrien16/CUNY_DATA_624/tree/master/Project_Two}{Pretend I'm a working link to Individual Work}
\end{compactitem}

\hypertarget{data-exploration}{%
\chapter{Data Exploration}\label{data-exploration}}

The beverage dataset includes 2,571 cases, 32 predictor variables, and a
single response variable. One of these predictor variables (Brand Code)
is categorical with four levels - A through D; for the purpose of our
analysis we interpreted these to represent four distinct beverage
brands.

While we found missing observations in both response and predictor
variables, in our assessment the extent of NAs did not suggest a
systemic issue in measurement or recording that imputing values could
not remedy. For context: - The response variable (PH) is missing a total
of four observations (\textless{} 1\%). - Most (30) predictor variables
are missing at least one observation, but only eleven are missing more
than 1\% of total cases and only three are missing more than 2\% of
total cases. These are: 1. MFR (continuous, 8.2\%) 2. BrandCode
(categorical, 4.7\%) 3. and FillerSpeed (continuous, 2.2\%)

{[}CONTENT EDITORS: DO WE STILL WANT TO CREATE MISSING DATA TABLE? IF
SO, MISSINGDATA OBJECT NEEDS TO BE REBUILT IN MODEL\_PREP.R{]}

\hypertarget{response-variable}{%
\section{Response Variable}\label{response-variable}}

{[}Density Plots{]}

Our target variable pH, is a continuous variable. pH is the inverse
logarithmic scaled measure of hydrogen ions in solutions and reflects
how acidic or basic a water-based solution is. Centered around a neutral
value of 7, pH ranges from highly acidic 0 and to highly alkaline at 14.

In total, the pH variable is approximately normally distributed,
centered around 8.546 (i.e.~slightly base), with some negative skew /
outliers. When evaluated by BrandCode: - A (293 observations) appears to
be multimodal and have the most outliers, with a mean slightly lower
than the aggregate (8.495) - B (1293 observations) appears to be bimodal
with a number of outliers, as well as a mean nearest the aggregate
(8.562) - C (304 observations) appears to be bimodal and is the most
acid (8.419) - D (615 observations) is the most normal distribution and
also has the highest alkalinity (8.603) - Missing Values

\hypertarget{predictor-variables}{%
\section{Predictor Variables}\label{predictor-variables}}

We examined the density of our variables to visualize the distribution
of the predictors. Many of these variables contain outliers and present
with a skewed distribution. The outliers fall outside the red-line
boundaries, and highlight which predictors have heavier tails.

The density plots also contain an overlay of the only categorical
indicator, \texttt{BrandCode}. This view shows us that some variables,
including \texttt{AlchRel}, \texttt{CarbRel}, \texttt{CarbVolume},
\texttt{HydPressure4}, and \texttt{Tempature}, are strongly influenced
by brand type.

{[}CONTENT EDITORS: DO WE STILL WANT TO CREATE THESE TABLES? IF SO,
OUTLIER\_WITH OBJECT NEEDS TO BE REBUILT IN MODEL\_PREP.R{]}

\textbf{{[} is this going to be evidenced by our graphs? As no predictor
variable shows a particularly pronounced monotonic linear relationship
with response, a non-linear approach to modeling seems warranted. {]}}

{[}FIGURE SUCH AND SUCH{]} helps to further visualize the effect
\texttt{BrandCode} has on our predictor and \texttt{pH} values. For
example, \texttt{AlchRel} shows distinct \texttt{BrandCode} groupings.
Other variables, such as \texttt{PSCO2}, \texttt{BowlSetpoint},
\texttt{MinFlow}, and \texttt{PressureSetup} show unique features likely
related to system processes.

\hypertarget{correlations}{%
\subsection{Correlations}\label{correlations}}

The plot below shows that BallingAlch, RelBalling, LvlDensityCarb,
RelBrand, CodeDCarb., VolumeCarb, Pressure are all highly correlated
with each other, but not particularly highly correlated with the
outcome, pH variable. They are all 25\% or less correlated with pH, as
pH is with most other variables both positive and negative. No extreme
heroics were necessary here, despite their being some variables which
are highly correlated with each other, because they were sufficiently
uncorrelated with the outcome variable and it is not clear how much
these are influenced by Brand Code such that removing some may
preferentially bias certain brands.

{[}NEW CHART WITH 2 Sets of Labels?{]}

\hypertarget{data-preparation}{%
\chapter{Data Preparation}\label{data-preparation}}

Preparing the data was the most discussed and influential part of our
modeling process. It was clear from early on that in order to build a
useful model with such a narrow range of expected pH values, how we
groomed our data and the decisions we made would likely be as or more
influential than the model we ultimately chose.

\textbf{Train/Test Splits:}

Prior to all pre-processing, we divided the production dataset using an
80/20 split to create a train and test sets.

All training models incorporated k-folds cross-validation set at 10
folds to protect against overfitting the data. We set up unique model
tuning grids to find the optimal parameters for each regression type to
ensure the highest accuracy within our predictions.

For both KNN and SVM models a grid of seeds was created from our
original seed to ensure that out repeated cross validation would be
repeateable. The same seeds were used in both the SVM and KNN.

{]}

\textbf{Data Imputation:}

Missing values are imputed using the \texttt{caret} package so that the
same range of imputed values could be applied to the test and validation
sets without confounding our training data and a bagging algorithm was
used to impute all continuous variables.

Because we were convinced that the 'brand variable \texttt{BrandCode}
may be one of the strongest predictors of pH, after much discussion, we
decided not to impute the Brand Code variable, so that each of the
observations with a known brand would be more accurately described by
the other variables relative to pH.

Instead the missing labels were replaces with Unknown and the variables
were converted to dummies of 0 and 1 to ensure that all modeling methods
would be able to consider Brand Code.

Test data is imputed with the same model, with that target variable
\texttt{PH} removed from the set.

\textbf{Pre-Processing:}

Most of the models concidered in our modeling process require scaling
and centering, so we included this in our prepartions. Although, only
one variable showed near-zero variance, Hyde.Pressure\_1 we opted to
remove it from all models during preprocessing and likewise applied
Box-Cox conversions to the data to compensate for andy skews and
non-normal modaliteis in the variables which might confound our models.
Again, the preprocessing model was saved so that the test and validation
sets could be consistenly transformed using caret's predict method.

\hypertarget{modeling}{%
\chapter{Modeling}\label{modeling}}

We assessed the effectiveness of more than ten different non-linear
regression models in our exploratory process. We settled on four models
that exhibited the most favorable test metrics, tuned those models, and
then chose the best performing model of that set to use in our final
analyses (all performance results from the other five are included in
{[}TABLE BLAH BLAH{]} in {[}APPENDIX BLAH BLAH{]}).

{[}BETHANY: INSERT SIDE-BY-SIDE TRAINING / TESTING METRICS FOR
PREPRCESSED MODELS (SET 2) HERE{]}.

\begin{itemize}
\tightlist
\item
  Model 1: Support Vector Machines Regression
\item
  Model 2: Cubist Tree Regression
\item
  Model 3: Multivariate Adaptive Regression Splines Regression
\item
  Model 4: Random Forest Regression
\end{itemize}

\hypertarget{model-performance}{%
\chapter{Model Performance}\label{model-performance}}

\begin{itemize}
\tightlist
\item
  Set1 = Caret: bagImputed; no additional pre-processing\\
\item
  Set2 = Caret: bagImputed; PreP
  \texttt{method=c(\textquotesingle{}center\textquotesingle{},\ \textquotesingle{}scale\textquotesingle{},\ \textquotesingle{}nzv\textquotesingle{},\ \textquotesingle{}BoxCox\textquotesingle{})}
\end{itemize}

\textbf{Train Performance:}

\begin{table}[H]

\caption{\label{tab:unnamed-chunk-6}Train1 Performance}
\centering
\fontsize{8}{10}\selectfont
\begin{tabular}[t]{rrrrl}
\toprule
MAPE & RMSE & RSquared & MAE & Method\\
\midrule
\rowcolor{gray!6}  1.0948 & 0.5916 & 0.6757 & 0.4319 & rf\\
1.1629 & 0.5500 & 0.6979 & 0.3938 & cubist\\
\rowcolor{gray!6}  1.2664 & 0.6985 & 0.5165 & 0.5037 & svmRadial\\
1.5170 & 0.6921 & 0.5261 & 0.5153 & earth\\
\bottomrule
\end{tabular}
\end{table}

\begin{table}[H]

\caption{\label{tab:unnamed-chunk-6}Train2 Performance}
\centering
\fontsize{8}{10}\selectfont
\begin{tabular}[t]{rrrrl}
\toprule
MAPE & RMSE & RSquared & MAE & Method\\
\midrule
\rowcolor{gray!6}  0.0081 & 0.0965 & 0.6904 & 0.0688 & cubist\\
0.0088 & 0.1030 & 0.6735 & 0.0751 & rf\\
\rowcolor{gray!6}  0.0104 & 0.1224 & 0.5042 & 0.0883 & svmRadial\\
0.0112 & 0.1265 & 0.4690 & 0.0954 & earth\\
\bottomrule
\end{tabular}
\end{table}

\textbf{Test Accuracy:}

\begin{table}[H]

\caption{\label{tab:unnamed-chunk-7}Test1 Performance}
\centering
\fontsize{8}{10}\selectfont
\begin{tabular}[t]{rrrrl}
\toprule
MAPE & RMSE & Rsquared & MAE & Method\\
\midrule
\rowcolor{gray!6}  0.4930 & 0.2234 & 0.9539 & 0.1578 & cubist\\
0.5351 & 0.3265 & 0.9568 & 0.2466 & rf\\
\rowcolor{gray!6}  0.9987 & 0.6076 & 0.6385 & 0.4101 & svmRadial\\
1.6688 & 0.7260 & 0.5226 & 0.5269 & earth\\
\bottomrule
\end{tabular}
\end{table}

\begin{table}[H]

\caption{\label{tab:unnamed-chunk-7}Test2 Performance}
\centering
\fontsize{8}{10}\selectfont
\begin{tabular}[t]{rrrrl}
\toprule
MAPE & RMSE & Rsquared & MAE & Method\\
\midrule
\rowcolor{gray!6}  0.0029 & 0.0367 & 0.9586 & 0.0250 & cubist\\
0.0037 & 0.0436 & 0.9596 & 0.0311 & rf\\
\rowcolor{gray!6}  0.0084 & 0.1064 & 0.6307 & 0.0717 & svmRadial\\
0.0107 & 0.1193 & 0.5257 & 0.0908 & earth\\
\bottomrule
\end{tabular}
\end{table}

{[}BETHANY: EACH OF US SHOULD WRITE BULLETS WITH REASONS TO CHOOSE THIS
MODEL BY SAT EVENING 12/7 - BETHANY WILL FLESH OUT{]}

\hypertarget{model-selection-considerations}{%
\section{Model Selection
Considerations}\label{model-selection-considerations}}

{[}BETHANY: NEED TO PICK OUR FIRST CHOICE MODEL, THINK IS SHOULD BE
VARIMP-ABLE SO WE CAN USE THAT IN INTERPRETATION / CONCLUSIONS{]}

\hypertarget{model-1-support-vector-machines-svm-regression}{%
\section{Model 1: Support Vector Machines (SVM)
Regression}\label{model-1-support-vector-machines-svm-regression}}

{[}BETHANY TO WORDSMITH RATIONALE FOR USING SVM MODEL{]} Support vector
machine (SVM) regression with a radial bias functin kernel is a
promising choice for predicting beverage pH because it excels when
working with data which may not be linearly separable, which comes into
play with this data specifically because pH is non-linear.

Although less efficient than the k-nearest neighbor and Multiple
Adaptive Regression Splines to train, the SVM provided robust final
model using a radial kernel with a cost of 10, passed as the tune length
settling on \(\sigma = 0.020\) and \(cost = 8\) returning a
\(RMSE = 0.1127\)

\begin{Shaded}
\begin{Highlighting}[]
\CommentTok{# remove echo/eval later [BETHANY: ADD IN VARIMP /}
\CommentTok{# PERFORMANCE CHART FOR SVM AS ALIGNED WITH GROUP]}
\end{Highlighting}
\end{Shaded}

\hypertarget{model-2-cubist-tree-regression}{%
\section{Model 2: Cubist Tree
Regression}\label{model-2-cubist-tree-regression}}

{[}JEREMY: CONDENSING BELOW WITH RATIONALE FOR USING MODEL IN BULLET
FORM BY EVENING SAT 12/7 - BETHANY WILL WORDSMITH{]}

· Cubist regression models provide a balance between predictive accuracy
interpretability · For a continuous response variable, Cubist models
functions like as piecewise linear model · The model creating rules
(which can overlap) to subset the data and then regression models to
each subset to arrive at a prediction. · They can also integrate
instance-based, nearest neighbors ensembling and boosting using
committees. · Based on cross-validation {[}ASSUMING THIS IS STILL
CORRECT{]} and grid search across hyper-parameters, we found best RMSE
with an instance-based model tuned to 5 neighbors and 50 committees.
Based on cross-validation and a grid search across hyper-parameters, we
found the best RMSE performance with an instance-based model that
factoring in many neighbors built on non-pre-processed training data.

\hypertarget{model-3-multivariate-adaptive-regression-splines-mars-regression}{%
\section{Model 3: Multivariate Adaptive Regression Splines (MARS)
Regression}\label{model-3-multivariate-adaptive-regression-splines-mars-regression}}

Multivariate regression splines (MARS) are more flexible about
relationships between preditors and the outcome variable than linear
regression models yet maintain their ease of interpretation.

MARS models also perform well without major pre processing steps with
reasonable bias-variance trade-off and are computationally efficient as
well as optimized work on very large data sets efficiently.

\hypertarget{model-4-random-forest-regression}{%
\section{Model 4: Random Forest
Regression}\label{model-4-random-forest-regression}}

{[}ANDY: PLEASE ADD CONCISE BULLETS WITH RATIONALE FOR USING RF MODEL BY
EVENING SAT 12/7 - BETHANY WILL WORDSMITH{]}

The optimal parameters for model was mtry = 31 and ntree = 2500. MAPE is
\textbf{r s\$MAPE} where as top 3 important predictors are
\texttt{MnfFlow}, \texttt{BrandCode} and \texttt{PressureVacuum} for
\%incMSE and \texttt{MnfFlow}, \texttt{BrandCode} and
\texttt{OxygenFiller} for IncNodePurity. Unlike \texttt{PLS},
\texttt{Random\ Forest} can produce 2 different variable importance
plots.

The first graph shows how much MSE would increase if a variable is
assigned with values by random permutation. The second plot is based on
\texttt{node\ purity} which is measured by the difference between RSS
before and after the split on that variable (\texttt{Gini\ Index}). In
short, each graph shows how much MSE or Impurity increases when each
variable is randomly permuted.

\hypertarget{interpretation}{%
\chapter{Interpretation}\label{interpretation}}

{[}BETHANY MAKING MAGIC HAPPEN WITH APPROPRIATE VARIMP GRAPH ONCE FINAL
MODEL SELECTED{]}

\hypertarget{conclusion}{%
\chapter{Conclusion}\label{conclusion}}

{[}BETHANY CREATING NEXT STEPS FOR PRODUCTION PROCESS BASED ON FINAL
MODEL{]}

\hypertarget{Appendix}{%
\chapter*{Appendix}\label{Appendix}}
\addcontentsline{toc}{chapter}{Appendix}

\textbf{Code}

\textbf{Data Dictionary}

\textbf{Exploratory Plots and List Models}

\hypertarget{citations}{%
\chapter{Citations}\label{citations}}

Shelton, Robert B. ``PH Values Of Common Drinks.'' Robert B. Shelton,
DDS MAGD Dentist Longview Texas, 2019,
www.sheltondentistry.com/patient-information/ph-values-common-drinks/.

Cubist Model Background: \url{https://www.rulequest.com/cubist-win.html}
Cubist Model Overview:
\url{https://static1.squarespace.com/static/51156277e4b0b8b2ffe11c00/t/56e3056a3c44d8779a61988a/1457718645593/cubist_BRUG.pdf}
Cubist Model Mechanics:
\url{http://ftp.uni-bayreuth.de/math/statlib/R/CRAN/doc/vignettes/caret/caretTrain.pdf}{]}

(\url{https://en.wikipedia.org/wiki/PH}).


\end{document}
