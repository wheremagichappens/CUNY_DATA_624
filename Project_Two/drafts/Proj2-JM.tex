\documentclass[]{report}
\usepackage{lmodern}
\usepackage{amssymb,amsmath}
\usepackage{ifxetex,ifluatex}
\usepackage{fixltx2e} % provides \textsubscript
\ifnum 0\ifxetex 1\fi\ifluatex 1\fi=0 % if pdftex
  \usepackage[T1]{fontenc}
  \usepackage[utf8]{inputenc}
\else % if luatex or xelatex
  \ifxetex
    \usepackage{mathspec}
  \else
    \usepackage{fontspec}
  \fi
  \defaultfontfeatures{Ligatures=TeX,Scale=MatchLowercase}
\fi
% use upquote if available, for straight quotes in verbatim environments
\IfFileExists{upquote.sty}{\usepackage{upquote}}{}
% use microtype if available
\IfFileExists{microtype.sty}{%
\usepackage{microtype}
\UseMicrotypeSet[protrusion]{basicmath} % disable protrusion for tt fonts
}{}
\usepackage{hyperref}
\hypersetup{unicode=true,
            pdftitle={PROJECT 2: PREDICTING PH},
            pdfauthor={Juliann McEachern},
            pdfborder={0 0 0},
            breaklinks=true}
\urlstyle{same}  % don't use monospace font for urls
\usepackage{graphicx,grffile}
\makeatletter
\def\maxwidth{\ifdim\Gin@nat@width>\linewidth\linewidth\else\Gin@nat@width\fi}
\def\maxheight{\ifdim\Gin@nat@height>\textheight\textheight\else\Gin@nat@height\fi}
\makeatother
% Scale images if necessary, so that they will not overflow the page
% margins by default, and it is still possible to overwrite the defaults
% using explicit options in \includegraphics[width, height, ...]{}
\setkeys{Gin}{width=\maxwidth,height=\maxheight,keepaspectratio}
\IfFileExists{parskip.sty}{%
\usepackage{parskip}
}{% else
\setlength{\parindent}{0pt}
\setlength{\parskip}{6pt plus 2pt minus 1pt}
}
\setlength{\emergencystretch}{3em}  % prevent overfull lines
\providecommand{\tightlist}{%
  \setlength{\itemsep}{0pt}\setlength{\parskip}{0pt}}
\setcounter{secnumdepth}{0}

%%% Use protect on footnotes to avoid problems with footnotes in titles
\let\rmarkdownfootnote\footnote%
\def\footnote{\protect\rmarkdownfootnote}

%%% Change title format to be more compact
\usepackage{titling}

% Create subtitle command for use in maketitle
\providecommand{\subtitle}[1]{
  \posttitle{
    \begin{center}\large#1\end{center}
    }
}

\setlength{\droptitle}{-2em}

  \title{PROJECT 2: PREDICTING PH}
    \pretitle{\vspace{\droptitle}\centering\huge}
  \posttitle{\par}
    \author{Juliann McEachern}
    \preauthor{\centering\large\emph}
  \postauthor{\par}
      \predate{\centering\large\emph}
  \postdate{\par}
    \date{10 December 2019}

% set plain style for page numbers
\usepackage[margin=1in]{geometry}
\usepackage{fancyhdr}
\pagestyle{fancy}
\fancyhead[LE,RO]{\textbf{Group 2}}
\fancyhead[RE,LO]{\textbf{Project 2: Predicting PH}}
\raggedbottom
\setlength{\parskip}{1em}

% change font
\usepackage{fontspec}
\setmainfont{Arial}

% format titles 
\usepackage{xcolor}
\usepackage{sectsty}
\usepackage{etoolbox}
\usepackage{titling}
\definecolor{prettyblue}{RGB}{84, 144, 240}
\definecolor{bluegray}{RGB}{98, 107, 115}
\pretitle{\begin{center}\Huge\color{prettyblue}\textbf}
\posttitle{\par\LARGE\color{gray}DATA 624 - Predictive Analytics\linebreak Group 2\end{center}}
\preauthor{\begin{center}\large\textbf{Group Members:}\linebreak\textit}
\postauthor{\end{center}}

% Format chapter output
\usepackage{titlesec}
\titleclass{\part}{top}
\titleclass{\chapter}{straight}
\titleformat{\chapter}
  {\normalfont\color{prettyblue}\LARGE\bfseries}{\thechapter}{1em}{}
\titlespacing*{\chapter}{0pt}{3.5ex plus 1ex minus .2ex}{2.3ex plus .2ex}


% create color block quotes
\usepackage{tcolorbox}
\newtcolorbox{myquote}{colback=purple!05!white, colframe=purple!75!black}
\renewenvironment{quote}{\begin{myquote}}{\end{myquote}}

% kable 
\usepackage{tabu}


% multicolumn
\usepackage{multicol}

% bullets
\newenvironment{tight_enumerate}{
\begin{enumerate}
  \setlength{\itemsep}{0pt}
  \setlength{\parskip}{0pt}
  }{\end{enumerate}}
  
\newenvironment{tight_itemize}{
\begin{itemize}
  \setlength{\topsep}{0pt}
  \setlength{\itemsep}{0pt}
  \setlength{\parskip}{0pt}
  \setlength{\parsep}{0pt}
  }{\end{itemize}}

\usepackage{paralist}

%hyperlink
\usepackage{hyperref}
\hypersetup{
    colorlinks=true,
    linkcolor=bluegray,
    filecolor=magenta,      
    urlcolor=cyan}

\usepackage{graphicx}
\usepackage{wrapfig}
\usepackage{booktabs}
\definecolor{yale}{RGB}{13,77,146}
\usepackage[font={color=yale,bf,scriptsize},figurename=Fig.,belowskip=0pt,aboveskip=0pt]{caption}
\usepackage{floatrow}
\floatsetup[figure]{capposition=above}
\floatsetup[table]{capposition=above}
\setlength{\abovecaptionskip}{1pt}
\setlength{\belowcaptionskip}{1pt}
\setlength{\textfloatsep}{2pt plus 0.5pt minus 0.5pt}
\setlength{\intextsep}{2pt plus 0.5pt minus 0.5pt}
\usepackage{booktabs}
\usepackage{longtable}
\usepackage{array}
\usepackage{multirow}
\usepackage{wrapfig}
\usepackage{float}
\usepackage{colortbl}
\usepackage{pdflscape}
\usepackage{tabu}
\usepackage{threeparttable}
\usepackage{threeparttablex}
\usepackage[normalem]{ulem}
\usepackage{makecell}
\usepackage{xcolor}

\begin{document}
\maketitle

{
\setcounter{tocdepth}{1}
\tableofcontents
}
\thispagestyle{empty}
\newpage
\clearpage
\pagenumbering{arabic}

\hypertarget{intro}{%
\chapter*{Introduction}\label{intro}}
\addcontentsline{toc}{chapter}{Introduction}

This project is designed to evaluate production data from a beverage
manufacturing company. Our assignment is to predict \texttt{PH}, a Key
Performance Indicator (KPI), with a high degree of accuracy through
predictive modeling. After thorough examination, we approached this task
by splitting the provided data into training and test sets. We evaluated
several models on this split and found that
\textbf{what-ever-worked-best} method yielded the best results.

Each group member worked individually to create their own solution. We
built our final submission by collaboratively evaluating and combining
each others' approaches. Our introduction should further outline
individual responsibilities. For example, \textbf{so-and-so} was
responsible for \textbf{xyz task}.

For replication and grading purposes, we made our code avaliable in the
appendix section. This code, along with the provided data, score-set
results, and individual contributions, can also be accessed through our
group github repository:

\begin{compactitem}
  \item \href{https://github.com/JeremyOBrien16/CUNY_DATA_624/tree/master/Project_Two}{Pretend I'm a working link to R Source Code}
  \item \href{https://github.com/JeremyOBrien16/CUNY_DATA_624/tree/master/Project_Two}{Pretend I'm a working link to Provided Data}
  \item \href{https://github.com/JeremyOBrien16/CUNY_DATA_624/tree/master/Project_Two}{Pretend I'm a working link to Excel Results}
  \item \href{https://github.com/JeremyOBrien16/CUNY_DATA_624/tree/master/Project_Two}{Pretend I'm a working link to Individual Work}
\end{compactitem}

\hypertarget{data-exploration}{%
\chapter{Data Exploration}\label{data-exploration}}

The beverage manufacturing production dataset contained 33
columns/variables and 2,571 rows/cases. In our initial review, we found
that the response variable, \texttt{PH}, had four missing observations.

We also identified that 94\% of the predictor variables had missing data
points. Despite this high occurance, the NA values in the majority of
these predictors accounted for less than 1\% of the total observations.
Only eleven variables were missing more than 1\% of data.

\begin{table}[H]

\caption{\label{tab:unnamed-chunk-1}Variables with Highest Frequency of NA Values}
\centering
\fontsize{8}{10}\selectfont
\begin{tabular}{lrrrrrrrrrrr}
\toprule
\textbf{ } & \textbf{MFR} & \textbf{BrandCode} & \textbf{FillerSpeed} & \textbf{PCVolume} & \textbf{PSCCO2} & \textbf{FillOunces} & \textbf{PSC} & \textbf{CarbPressure1} & \textbf{HydPressure4} & \textbf{CarbPressure} & \textbf{CarbTemp}\\
\midrule
\rowcolor{gray!6}  n & 212.0 & 120.0 & 57.0 & 39.0 & 39.0 & 38.0 & 33.0 & 32.0 & 30.0 & 27.0 & 26\\
\% & 8.2 & 4.7 & 2.2 & 1.5 & 1.5 & 1.5 & 1.3 & 1.2 & 1.2 & 1.1 & 1\\
\bottomrule
\end{tabular}
\end{table}

\hypertarget{response-variable}{%
\section{Response Variable}\label{response-variable}}

\begin{wrapfigure}{r}{0.5\textwidth}[H]

\hfill{}\includegraphics[width=1\textwidth]{Proj2-JM_files/figure-latex/unnamed-chunk-2-1} 

\caption{Distribution of Response Variable: pH}\label{fig:unnamed-chunk-2}
\end{wrapfigure}

Understanding the influence pH has on our predictors is key to building
an accurate predictive model. pH is a measure of acidity/alkalinity that
must conform in a critical range. The value of pH ranges from 0 to 14,
where 0 is acidic, 7 is neutral, and 14 is basic.

Figure 1.1 shows that our response distribution follows a somewhat
normal pattern and is centered around 8.5. The histogram for \texttt{pH}
is bimodal in the aggregate, but varies by brand. The boxplot view
allows us to better visualize the effect outliers have on the skewness
within our target variable.

Brand A has a negatively skewed, multimodal distribution, which could be
suggestive of several distinct underlying response patterns or a higher
degree of variation in \texttt{pH} response for this brand. The density
plot and histogram for Brand B show two bimodal peaks with a slight
positive skew. These peaks indicate that this brand has two distinct
response values that occur more frequently. The distribution for Brand C
and D are both more normal, with a slight negative skew. Brand D has the
highest median \texttt{pH} value and Brand C has the lowest. Brand C
also appears to have the largest spread of \texttt{pH} values.

\hypertarget{predictor-variables}{%
\section{Predictor Variables}\label{predictor-variables}}

Many of our predictors also contain outliers and have a skewed
distribution. The boxplots below help us visualize this spread of our
numeric predictor variables.

\begin{figure}
\centering
\includegraphics{Proj2-JM_files/figure-latex/unnamed-chunk-3-1.pdf}
\caption{Box-Plot Distribution of Numeric Predictor Variables}
\end{figure}

We examined the predictor variables with outliers in a scatterplot
against our target, \texttt{pH} to better understand predictor and
response relationship. The outliers, highlighted in blue, further show
which predictors have a heavy-tail distribution. We can also identify
many variables with strong outlier patterns, suggesting a high degree of
variability within certain measurements.

\begin{figure}
\centering
\includegraphics{Proj2-JM_files/figure-latex/unnamed-chunk-4-1.pdf}
\caption{Response\textasciitilde{}Predictor Scatterplots}
\end{figure}

For example, \texttt{AirPressurer} shows a very distinct, bifurcated
pattern. This variable has a clear split between normal and extreme
values. \texttt{MFR} also shows an interesting pattern. The outliers
have a weak, negative linear relationship with \texttt{pH}, but the
non-outliers have no linear relationship and follow a straight, vertical
line.

\hypertarget{data-preparation}{%
\chapter{Data Preparation}\label{data-preparation}}

Decision models trees are robust against the affect of correlated
variables, outliers, and missing values. We applied different
tranformation to properly evaluate our three model types.

Data was divided using an 80/20 split to create a train and test set.
All models incorporated k-folds cross-validation set at 10 folds to
protect against overfitting the data.

\hypertarget{data-imputation}{%
\section{Data Imputation}\label{data-imputation}}

We choose to drop the complete cases of all \texttt{pH} observations
with null data in the target as they accounted for such a small
proportion (\textless{} 0.002\%) of the observations. Doing such
increased our non-linear modeling accuracy measures. For our predictor
variables, we applied a Multiple Imputation by Chained Equations (MICE)
algorthim to predict the missing data using sequential regression. This
method filled in all incomplete cases, including \texttt{BrandCode}, our
one unordered categorical variable.

\emph{We can also use this same approach to handle outliers (linear
model) by setting their value to \texttt{NA} and predicticting a value
within the expected range.}

\hypertarget{pre-processing}{%
\section{Pre-Processing}\label{pre-processing}}

\hypertarget{correlation}{%
\section{Correlation}\label{correlation}}

We examined the relationship between our numeric predictors and found
that 9 of the variables appear heavily related, with correlation values
exceeding \(\pm{0.75}\). The full correlation matrix can be viewed in
the appendix section. \emph{Revisit section to add more text}

\begin{table}[H]

\caption{\label{tab:unnamed-chunk-6}Highly Correlated Predictors}
\fontsize{8}{10}\selectfont
\begin{tabular}{lll>{\bfseries}llll}
\toprule
\textbf{V1} & \textbf{V2} & \textbf{COR} & \textbf{|} & \textbf{V1  } & \textbf{V2 } & \textbf{COR }\\
\midrule
\rowcolor{gray!6}  AlchRel & BallingLvl & 0.93 & | & CarbVolume & BallingLvl & 0.78\\
AlchRel & CarbRel & 0.84 & | & CarbVolume & Density & 0.76\\
\rowcolor{gray!6}  Balling & BallingLvl & 0.98 & | & Density & Balling & 0.96\\
Balling & AlchRel & 0.92 & | & Density & BallingLvl & 0.95\\
\rowcolor{gray!6}  Balling & CarbRel & 0.82 & | & Density & AlchRel & 0.90\\
\addlinespace
CarbPressure & CarbTemp & 0.81 & | & Density & CarbRel & 0.82\\
\rowcolor{gray!6}  CarbRel & BallingLvl & 0.84 & | & FillerLevel & BowlSetpoint & 0.95\\
CarbVolume & CarbRel & 0.79 & | & FillerSpeed & MFR & 0.93\\
\rowcolor{gray!6}  CarbVolume & Balling & 0.78 & | & HydPressure2 & HydPressure3 & 0.92\\
CarbVolume & AlchRel & 0.78 & | & MnfFlow & HydPressure3 & 0.76\\
\bottomrule
\end{tabular}
\end{table}

\emph{Test the effect of pre-processing methods to maximize the success
of our tree and non-tree models. Not currently adding data
transformations but may revist: ie. scale data for PLS. }

For linear models, we removed the predictor \texttt{HydPressure1} as it
contained near-zero variance. \texttt{HydPressure3}, \texttt{Balling},
\texttt{BallingLvl}, \texttt{FillerSpeed}, \texttt{FillerLevel}, and
\texttt{Density} were also removed due to large absolute correlations
with other variables.

\hypertarget{predictive-modeling}{%
\chapter{Predictive Modeling}\label{predictive-modeling}}

We modeled the data using linear, non-linear, and tree-based regression.
In this section, we discuss the attempts from our best model performance
for each regression type.

\hypertarget{linear-regression}{%
\section{Linear Regression}\label{linear-regression}}

multicollineraty

Explain text here. Text text text text text text text text text text
text text text text text text text text text text text text text text
text text text text text text text text text text text text text text
text text text text text text text text text text text text text text
text text text text text text text text text.

\hypertarget{non-linear-regression}{%
\section{Non-Linear Regression}\label{non-linear-regression}}

Explain text here. Text text text text text text text text text text
text text text text text text text text text text text text text text
text text text text text text text text text text text text text text
text text text text text text text text text text text text text text
text text text text text text text text text.

\hypertarget{tree-based-regression}{%
\section{Tree-Based Regression}\label{tree-based-regression}}

\textbf{MARS CV RMSE:}
\includegraphics{Proj2-JM_files/figure-latex/unnamed-chunk-7-1.pdf}

\textbf{Train Accuracy: }

\begin{table}[H]
\centering\begingroup\fontsize{8}{10}\selectfont

\begin{tabular}{l|r|r|r|r}
\hline
  & MARS1\_Train & MARS2\_Train & MARS1\_Test & MARS2\_Test\\
\hline
\rowcolor{gray!6}  RMSE & 0.1223 & 0.1230 & 0.1179 & 0.1269\\
\hline
Rsquared & 0.5023 & 0.5034 & 0.5214 & 0.4577\\
\hline
\rowcolor{gray!6}  MAE & 0.0938 & 0.0934 & 0.0882 & 0.0897\\
\hline
MAPE & 0.0105 & 0.0102 & 0.0103 & 0.0105\\
\hline
\end{tabular}
\endgroup{}
\end{table}

Variable Importance:

\begin{wraptable}{l}{0pt}[H]

\caption{\label{tab:unnamed-chunk-9}Comparison of MARS Models Var Importance}
\centering
\fontsize{8}{10}\selectfont
\begin{tabular}{l|r|r|r}
\hline
Variable & MARS2 & MARS1 & diff\\
\hline
\rowcolor{gray!6}  MnfFlow & 100.00 & 100.00 & 0.00\\
\hline
BrandCodeC & 78.60 & 77.38 & -1.21\\
\hline
\rowcolor{gray!6}  PCVolume & 78.60 & 16.54 & -62.06\\
\hline
HydPressure2 & 78.60 & 61.28 & -17.31\\
\hline
\rowcolor{gray!6}  AirPressurer & 69.77 & 66.43 & -3.34\\
\hline
Temperature & 65.27 & 61.28 & -3.99\\
\hline
\rowcolor{gray!6}  MFR & 65.27 & 0.00 & -65.27\\
\hline
BrandCodeD & 59.59 & 0.00 & -59.59\\
\hline
\rowcolor{gray!6}  CarbPressure1 & 54.94 & 39.12 & -15.82\\
\hline
OxygenFiller & 54.94 & 22.29 & -32.66\\
\hline
\rowcolor{gray!6}  Usagecont & 51.02 & 33.70 & -17.32\\
\hline
BowlSetpoint & 47.93 & 42.65 & -5.28\\
\hline
\rowcolor{gray!6}  PressureVacuum & 44.96 & 49.93 & 4.98\\
\hline
HydPressure1 & 39.72 & 39.12 & -0.60\\
\hline
\rowcolor{gray!6}  CarbFlow & 36.19 & 0.00 & -36.19\\
\hline
BrandCodeB & 0.00 & 0.00 & 0.00\\
\hline
\rowcolor{gray!6}  CarbVolume & 0.00 & 0.00 & 0.00\\
\hline
FillOunces & 0.00 & 0.00 & 0.00\\
\hline
\rowcolor{gray!6}  CarbPressure & 0.00 & 0.00 & 0.00\\
\hline
CarbTemp & 0.00 & 0.00 & 0.00\\
\hline
\rowcolor{gray!6}  PSC & 0.00 & 0.00 & 0.00\\
\hline
PSCFill & 0.00 & 0.00 & 0.00\\
\hline
\rowcolor{gray!6}  PSCCO2 & 0.00 & 0.00 & 0.00\\
\hline
FillPressure & 0.00 & 0.00 & 0.00\\
\hline
\rowcolor{gray!6}  HydPressure4 & 0.00 & 0.00 & 0.00\\
\hline
PressureSetpoint & 0.00 & 0.00 & 0.00\\
\hline
\rowcolor{gray!6}  CarbRel & 0.00 & 0.00 & 0.00\\
\hline
BallingLvl & 0.00 & 0.00 & 0.00\\
\hline
\end{tabular}
\end{wraptable}

\hypertarget{train}{%
\section{Train}\label{train}}

Train text.

\hypertarget{test}{%
\section{Test}\label{test}}

Test text.

\hypertarget{discussion}{%
\chapter{Discussion}\label{discussion}}

Eval text. The end.

\hypertarget{conclusion}{%
\chapter{Conclusion}\label{conclusion}}

sfasdfs

\hypertarget{Appendix}{%
\chapter*{Appendix}\label{Appendix}}
\addcontentsline{toc}{chapter}{Appendix}

\hypertarget{summary-statistics}{%
\section{Summary Statistics}\label{summary-statistics}}

\begin{table}[H]
\centering\begingroup\fontsize{8}{10}\selectfont

\begin{tabular}{lrrrrrrrrrrrrr}
\toprule
  & vars & n & mean & sd & median & trimmed & mad & min & max & range & skew & kurtosis & se\\
\midrule
\rowcolor{gray!6}  BrandCode* & 1 & 2451 & 2.5 & 1.0 & 2.0 & 2.5 & 0.0 & 1.0 & 4.0 & 3.0 & 0.4 & -1.1 & 0.0\\
CarbVolume & 2 & 2561 & 5.4 & 0.1 & 5.3 & 5.4 & 0.1 & 5.0 & 5.7 & 0.7 & 0.4 & -0.5 & 0.0\\
\rowcolor{gray!6}  FillOunces & 3 & 2533 & 24.0 & 0.1 & 24.0 & 24.0 & 0.1 & 23.6 & 24.3 & 0.7 & 0.0 & 0.9 & 0.0\\
PCVolume & 4 & 2532 & 0.3 & 0.1 & 0.3 & 0.3 & 0.1 & 0.1 & 0.5 & 0.4 & 0.3 & 0.7 & 0.0\\
\rowcolor{gray!6}  CarbPressure & 5 & 2544 & 68.2 & 3.5 & 68.2 & 68.1 & 3.6 & 57.0 & 79.4 & 22.4 & 0.2 & 0.0 & 0.1\\
\addlinespace
CarbTemp & 6 & 2545 & 141.1 & 4.0 & 140.8 & 141.0 & 3.9 & 128.6 & 154.0 & 25.4 & 0.2 & 0.2 & 0.1\\
\rowcolor{gray!6}  PSC & 7 & 2538 & 0.1 & 0.0 & 0.1 & 0.1 & 0.0 & 0.0 & 0.3 & 0.3 & 0.8 & 0.6 & 0.0\\
PSCFill & 8 & 2548 & 0.2 & 0.1 & 0.2 & 0.2 & 0.1 & 0.0 & 0.6 & 0.6 & 0.9 & 0.8 & 0.0\\
\rowcolor{gray!6}  PSCCO2 & 9 & 2532 & 0.1 & 0.0 & 0.0 & 0.0 & 0.0 & 0.0 & 0.2 & 0.2 & 1.7 & 3.7 & 0.0\\
MnfFlow & 10 & 2569 & 24.6 & 119.5 & 65.2 & 21.1 & 169.0 & -100.2 & 229.4 & 329.6 & 0.0 & -1.9 & 2.4\\
\addlinespace
\rowcolor{gray!6}  CarbPressure1 & 11 & 2539 & 122.6 & 4.7 & 123.2 & 122.5 & 4.4 & 105.6 & 140.2 & 34.6 & 0.1 & 0.1 & 0.1\\
FillPressure & 12 & 2549 & 47.9 & 3.2 & 46.4 & 47.7 & 2.4 & 34.6 & 60.4 & 25.8 & 0.5 & 1.4 & 0.1\\
\rowcolor{gray!6}  HydPressure1 & 13 & 2560 & 12.4 & 12.4 & 11.4 & 10.8 & 16.9 & -0.8 & 58.0 & 58.8 & 0.8 & -0.1 & 0.2\\
HydPressure2 & 14 & 2556 & 21.0 & 16.4 & 28.6 & 21.1 & 13.3 & 0.0 & 59.4 & 59.4 & -0.3 & -1.6 & 0.3\\
\rowcolor{gray!6}  HydPressure3 & 15 & 2556 & 20.5 & 16.0 & 27.6 & 20.5 & 13.9 & -1.2 & 50.0 & 51.2 & -0.3 & -1.6 & 0.3\\
\addlinespace
HydPressure4 & 16 & 2541 & 96.3 & 13.1 & 96.0 & 95.5 & 11.9 & 52.0 & 142.0 & 90.0 & 0.5 & 0.6 & 0.3\\
\rowcolor{gray!6}  FillerLevel & 17 & 2551 & 109.3 & 15.7 & 118.4 & 111.0 & 9.2 & 55.8 & 161.2 & 105.4 & -0.8 & 0.0 & 0.3\\
FillerSpeed & 18 & 2514 & 3687.2 & 770.8 & 3982.0 & 3920.0 & 47.4 & 998.0 & 4030.0 & 3032.0 & -2.9 & 6.7 & 15.4\\
\rowcolor{gray!6}  Temperature & 19 & 2557 & 66.0 & 1.4 & 65.6 & 65.8 & 0.9 & 63.6 & 76.2 & 12.6 & 2.4 & 10.2 & 0.0\\
Usagecont & 20 & 2566 & 21.0 & 3.0 & 21.8 & 21.3 & 3.2 & 12.1 & 25.9 & 13.8 & -0.5 & -1.0 & 0.1\\
\addlinespace
\rowcolor{gray!6}  CarbFlow & 21 & 2569 & 2468.4 & 1073.7 & 3028.0 & 2601.1 & 326.2 & 26.0 & 5104.0 & 5078.0 & -1.0 & -0.6 & 21.2\\
Density & 22 & 2570 & 1.2 & 0.4 & 1.0 & 1.2 & 0.1 & 0.2 & 1.9 & 1.7 & 0.5 & -1.2 & 0.0\\
\rowcolor{gray!6}  MFR & 23 & 2359 & 704.0 & 73.9 & 724.0 & 718.2 & 15.4 & 31.4 & 868.6 & 837.2 & -5.1 & 30.5 & 1.5\\
Balling & 24 & 2570 & 2.2 & 0.9 & 1.6 & 2.1 & 0.4 & -0.2 & 4.0 & 4.2 & 0.6 & -1.4 & 0.0\\
\rowcolor{gray!6}  PressureVacuum & 25 & 2571 & -5.2 & 0.6 & -5.4 & -5.3 & 0.6 & -6.6 & -3.6 & 3.0 & 0.5 & 0.0 & 0.0\\
\addlinespace
PH & 26 & 2567 & 8.5 & 0.2 & 8.5 & 8.6 & 0.2 & 7.9 & 9.4 & 1.5 & -0.3 & 0.1 & 0.0\\
\rowcolor{gray!6}  OxygenFiller & 27 & 2559 & 0.0 & 0.0 & 0.0 & 0.0 & 0.0 & 0.0 & 0.4 & 0.4 & 2.7 & 11.1 & 0.0\\
BowlSetpoint & 28 & 2569 & 109.3 & 15.3 & 120.0 & 111.3 & 0.0 & 70.0 & 140.0 & 70.0 & -1.0 & -0.1 & 0.3\\
\rowcolor{gray!6}  PressureSetpoint & 29 & 2559 & 47.6 & 2.0 & 46.0 & 47.6 & 0.0 & 44.0 & 52.0 & 8.0 & 0.2 & -1.6 & 0.0\\
AirPressurer & 30 & 2571 & 142.8 & 1.2 & 142.6 & 142.6 & 0.6 & 140.8 & 148.2 & 7.4 & 2.3 & 4.7 & 0.0\\
\addlinespace
\rowcolor{gray!6}  AlchRel & 31 & 2562 & 6.9 & 0.5 & 6.6 & 6.8 & 0.1 & 5.3 & 8.6 & 3.3 & 0.9 & -0.9 & 0.0\\
CarbRel & 32 & 2561 & 5.4 & 0.1 & 5.4 & 5.4 & 0.1 & 5.0 & 6.1 & 1.1 & 0.5 & -0.3 & 0.0\\
\rowcolor{gray!6}  BallingLvl & 33 & 2570 & 2.1 & 0.9 & 1.5 & 2.0 & 0.2 & 0.0 & 3.7 & 3.7 & 0.6 & -1.5 & 0.0\\
\bottomrule
\end{tabular}
\endgroup{}
\end{table}

\hypertarget{correlation-matrix}{%
\section{Correlation Matrix}\label{correlation-matrix}}

\begin{center}\includegraphics{Proj2-JM_files/figure-latex/unnamed-chunk-11-1} \end{center}


\end{document}
